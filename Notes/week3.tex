\ifx\PREAMBLE\undefined
\documentclass{article}
\usepackage[format = hang, font = bf]{caption}
\usepackage{graphicx}
\usepackage{array}
\usepackage{amsmath}
\usepackage{mathtools}
\usepackage{boxedminipage}
\usepackage{listings}
\usepackage{makecell}%diagonal line in table
\usepackage{float}%allowing forceful figure[H]
\usepackage{xcolor}
\usepackage{amsfonts}%allowing \mathbb{R}
\usepackage{alltt}
\usepackage{url}
\def\UrlBreaks{\do\A\do\B\do\C\do\D\do\E\do\F\do\G\do\H\do\I\do\J\do\K\do\L\do\M\do\N\do\O\do\P\do\Q\do\R\do\S\do\T\do\U\do\V\do\W\do\X\do\Y\do\Z\do\[\do\\\do\]\do\^\do\_\do\`\do\a\do\b\do\c\do\d\do\e\do\f\do\g\do\h\do\i\do\j\do\k\do\l\do\m\do\n\do\o\do\p\do\q\do\r\do\s\do\t\do\u\do\v\do\w\do\x\do\y\do\z\do\0\do\1\do\2\do\3\do\4\do\5\do\6\do\7\do\8\do\9\do\.\do\@\do\\\do\/\do\!\do\_\do\|\do\;\do\>\do\]\do\)\do\,\do\?\do\'\do+\do\=\do\#\do\-}
\usepackage[breaklinks = true]{hyperref}
\lstset{language = matlab, breaklines = true, tabsize = 2, numbers = left, extendedchars = false, basicstyle = {\ttfamily \footnotesize}, keywordstyle=\color{blue!70}, commentstyle=\color{red!70}, frame=shadowbox, rulesepcolor=\color{red!20!green!20!blue!20}, numberstyle={\color[RGB]{0,192,192}}}

\begin{document}
\fi
\section{Logistic regression}
\subsection{Hypothesis function and cost function}
Recall that in supervised learning, when the output is discretely valued, the problem is said to be a classification problem. When there are only two classes, the problem is called a logistic regression problem. We will illustrate how to solve such problem.

We use the hyphothesis function
\begin{equation}
h_{\theta}(x) = g(\theta^{\mathsf T}x) = \frac{1}{1+e^{-\theta^{\mathsf T}x}}
\end{equation}
where $g(z) = \frac{1}{1+e^{-z}}$ is called {\bf sigmoid function} or {\bf logistic function}.

$h_{\theta}(x)$ can be interpreted as the probablity that $y=1$ with input $x$, i.e.
$$h_{\theta}(x) = P(y = 1|x,\theta)$$

If $h_{\theta}(x) \ge 0.5$, we predict that $y=1$; otherwise ($h_{\theta}(x) < 0.5$) we predict $y=0$. This gives us the decision boundry $\theta^{\mathsf T}x=0$. 

Squared error cost function is no longer appropriate for logistic regression. We use the cost function
\begin{equation}
J(\theta) =-\frac{1}{m}\sum\limits_{i=1}^{m}\left(y^{(i)}\log(h_{\theta}(x^{(i)})) + (1-y^{(i)})\log(1-h_{\theta}(x^{(i)}))\right)
\end{equation}

The logistic regression problem can now be solved by minimize he cost function: $\min\limits_{\theta}J(\theta)$. Again we can apply gradient descent
\begin{equation}
\theta_j \coloneqq \theta_j - \frac{\alpha}{m}\sum\limits_{i=1}^{m}\left(h_{\theta}(x^{(i)})-y^{(i)}\right)x_j^{(i)}
\end{equation}

Some examples of more complex optimization algorithms: conjugate gradient, BFGS, L-BFGS. They do not require the manual selection of $\alpha$ and they are usually faster than gradient descent.

\subsection{One-vs-all classification}
When we have more than two classes, it is possible to solve the classification problem with the algorithm we have developped for logistic regression problems by using the socalled one-vs-all classification algorithm.

For each class $y=i$, we fit a hypothesis function (regression classifier) $h_{\theta}^{(i)}(x)$ that can be interpreted as the probability of $y=i$, i.e.
$$h_{\theta}^{(i)}(x) = P(y = i|x,\theta)$$

Equiped with these classifiers, when asked to predict the output for a certain $x$, we choose the class $i$ that maximizes $h_{\theta}^{(i)}(x)$.
\section{Regularization: solution to overfitting}
When too many features are involved in the regression, we could end up with the overfitting problem: the learned hypothesis fits the training set very well so that $J(\theta)\approx0$, but fails to make good prediction for new examples.

There are a few possible solutions to the overfitting problem.
\begin{enumerate}
\item Reduce number of features. Manual selection or using model selection algorithm, which will be covered later.
\item Regularization. All features are kept but magnitudes of $\theta_j$s are reduced. 
\end{enumerate}
\subsection{Cost function of regularization}
The principle of regularization is to reduce the magnitudes of parameters $\theta_j$. This can be achieved by adding a squared item to the cost function:
$$J(\theta) = J_0(\theta) + \frac{\lambda}{2m}\sum\limits_{j=1}^{n}\theta_j^2$$
By convention the added square item does not inclued $\theta_0$.
\subsection{Regularized linear regression}
For linear regression, we now have the new cost function
\begin{equation}
J(\theta) = \frac{1}{2m}\left[\sum\limits_{i=1}^{m}\left(h_{\theta}(x^{(i)}) - y^{(i)}\right)^2 + \lambda\sum\limits_{j=1}^{n}\theta_j^2\right]
\end{equation}
And the iteration equation in gradient descent becomes
\begin{equation}
\begin{split}
\theta_0 &\coloneqq \theta_0 - \frac{\alpha}{m}\sum\limits_{i=1}^{m}\left(h_{\theta}(x^{(i)}) - y^{(i)}\right)
x^{(i)}_0\\
\theta_j &\coloneqq \theta_j - \alpha\left[\frac{1}{m}\sum\limits_{i=1}^{m}\left(h_{\theta}(x^{(i)}) - y^{(i)}\right)
x^{(i)}_j + \frac{\lambda}{m}\theta_j\right]\\
&=\theta_j\left(1-\frac{\alpha\lambda}{m}\right)-\frac{\alpha}{m}\sum\limits_{i=1}^{m}\left(h_{\theta}(x^{(i)}) - y^{(i)}\right)
x^{(i)}_j, j = 1,2 \dots n
\end{split}
\end{equation}
Intuitively, the $\left(1-\frac{\alpha\lambda}{m}\right)$ factor reduces the magnitude of $\theta_j$ at each iteration.

The solution of the normal equation changes into
\begin{equation}
\theta = \left(X^{\mathsf T}X + \lambda
\begin{bmatrix}
0\\
 &1\\
 & &1\\
 & & &\ddots\\
 & & & &1\\	
\end{bmatrix}
\right)^{-1}X^{\mathsf T}y
\end{equation}
It can be proved that as long as $\lambda > 0$, the matrix to be inverted is invertible, which solves the problem of non-invertibility of $X^{\mathsf T}X$.
\subsection{Regularized logistic regression}
For logistic regression, we now have the cost function
\begin{equation}
J(\theta) = -\frac{1}{m}\sum\limits_{i=1}^{m}\left(y^{(i)}log(h_{\theta}(x^{(i)}))+(1-y^{(i)})log\left(1-h_{\theta}(x^{(i)})\right)\right) + \frac{\lambda}{2m}\sum\limits_{j=1}^{n}\theta_j^2
\end{equation}
And the iteration equation of gradient descent becomes
\begin{equation}
\begin{split}
\theta_0 &\coloneqq \theta_0 - \frac{\alpha}{m}\sum\limits_{i=1}^{m}\left(h_{\theta}(x^{(i)}) - y^{(i)}\right)
x^{(i)}_0\\
\theta_j &\coloneqq \theta_j - \alpha\left[\frac{1}{m}\sum\limits_{i=1}^{m}\left(h_{\theta}(x^{(i)}) - y^{(i)}\right)
x^{(i)}_j + \frac{\lambda}{m}\theta_j\right]\\
&=\theta_j\left(1-\frac{\alpha\lambda}{m}\right)-\frac{\alpha}{m}\sum\limits_{i=1}^{m}\left(h_{\theta}(x^{(i)}) - y^{(i)}\right)
x^{(i)}_j, j = 1,2 \dots n
\end{split}
\end{equation}
which is actually the same as that of linear regression.
\ifx\PREAMBLE\undefined
\end{document}
\fi